% Created 2015-02-20 Пт 06:18
\documentclass[presentation]{beamer}


%%% Packages.

\usepackage[utf8]{inputenc}
\usepackage[T1]{fontenc}
\usepackage{fixltx2e}
\usepackage{graphicx}
\usepackage{longtable}
\usepackage{float}
\usepackage{wrapfig}
\usepackage{rotating}
\usepackage[normalem]{ulem}
\usepackage{amsmath, amssymb}
\usepackage{textcomp}
\usepackage{marvosym}
\usepackage{wasysym}
\usepackage{hyperref}
\tolerance=1000
\usepackage[english, russian]{babel}
\usepackage[labelformat=empty]{caption}
\usepackage{subcaption}
\usepackage{color}
\let\Cross\relax
\let\Square\relax
\usepackage{bbding}
\usepackage{fancyvrb}


%%%

\graphicspath{{graphics/}}

\usetheme[height=20pt]{Rochester}


%%% Title.

\author{Артём Попцов}
\date{2016-09-01}

\title{GNU Privacy Guard}

\begin{document}

\maketitle


%%% TOC.

\begin{frame}{Содержание}
  \setcounter{tocdepth}{1}
  \tableofcontents
\end{frame}


%%%

\section{Введение}

\subsection{Введение}

\begin{frame}{Введение}
  Задачи, решаемые с помощью GnuPG:\newline
  \begin{itemize}
  \item Шифрование данных
    \begin{itemize}
    \item Защита данных от несанкционированного доступа
    \item Обеспечение приватности переписки\newline
    \end{itemize}
  \item Электронная подпись
    \begin{itemize}
    \item Контроль целостности данных
    \item Защита от изменений (подделки) данных
    \item Доказательство авторства документа
    \item Фиксирование невозможности отказа от авторства
    \end{itemize}
  \end{itemize}
\end{frame}


%%%

\subsection{Настройка GnuPG}

\begin{frame}[fragile]{Установка GnuPG}
  На большинстве дистрибутивов GNU/Linux GnuPG можно поставить из
  репозитория.  Пример для Ubuntu GNU/Linux:
\begin{Verbatim}[commandchars=\\\[\]]
\$ sudo apt-get install gnupg gnupg-agent
\end{Verbatim}
\vspace{5 mm}
Что и зачем ставим:
\begin{itemize}
\item \alert{gnupg} -- GNU privacy guard -- свободная замена PGP
\item \alert{gnupg-agent} -- GNU privacy guard -- менеджер паролей\newline
\end{itemize}

На Microsoft Windows можно воспользоваться \textbf{Gpg4win}
(\url{https://www.gpg4win.org/}).
\end{frame}


\section{Создание пары ключей}

\subsection{Создание пары ключей}

\begin{frame}[fragile]{Создание пары ключей -- 1}
  Пара ``закрытый ключ -- публичный ключ'' может быть создана
  следующей командой:
\begin{verbatim}
$ gpg --gen-key
\end{verbatim}
  При создании ключа необходимо:
  \begin{enumerate}
  \item Выбрать тип ключа:
\begin{Verbatim}[commandchars=\\\[\]]
Please select what kind of key you want:
   (1) RSA and RSA (default)
   (2) DSA and Elgamal
   (3) DSA (sign only)
   (4) RSA (sign only)
Your selection? [\color[red]1]
\end{Verbatim}
    \end{enumerate}
\end{frame}

\begin{frame}[fragile]{Создание пары ключей -- 2}
  \begin{enumerate}
    \setcounter{enumi}{1}
  \item Выбрать длину ключа:
\begin{Verbatim}[commandchars=\\\[\]]
RSA keys may be between 1024 and 4096 bits long.
What keysize do you want? (2048) [\color[red]2048]
Requested keysize is 2048 bits
\end{Verbatim}
  \item Выбрать ``срок годности'' ключа:
\begin{Verbatim}[commandchars=\\\[\]]
Please specify how long the key should be valid.
         0 = key does not expire
      <n>  = key expires in n days
      <n>w = key expires in n weeks
      <n>m = key expires in n months
      <n>y = key expires in n years
Key is valid for? (0) [\color[red]0]
Key does not expire at all
Is this correct? (y/N) [\color[red]y]
\end{Verbatim}
    \end{enumerate}
\end{frame}

\begin{frame}[fragile]{Создание пары ключей -- 3}
  \begin{enumerate}
    \setcounter{enumi}{3}
  \item Указать реальное имя владельца ключа:
\begin{Verbatim}[commandchars=\\\[\]]
GnuPG needs to construct a user ID to identify
  your key.

Real name: [\color[red]Vasiliy I. Pupkin]
\end{Verbatim}
  \item Указать адрес электронной почты владельца:
\begin{Verbatim}[commandchars=\\\[\]]
Email address: [\color[red]vip@example.ru]
\end{Verbatim}

  \item (Опционально) указать комментарий к ключу:
\begin{Verbatim}[commandchars=\\\[\]]
Comment:
\end{Verbatim}
    \end{enumerate}
\end{frame}

\begin{frame}[fragile]{Создание пары ключей -- 4}
  \begin{enumerate}
  \setcounter{enumi}{6}
\item Проверить указанную информацию и подтвердить её:
  \begin{Verbatim}[commandchars=\\\[\]]
You selected this USER-ID:
    "Vasiliy I. Pupkin <vip@example.ru>"

Change (N)ame, (C)omment, (E)mail or (O)kay/(Q)uit? [\color[red]O]
  \end{Verbatim}

\item Задать пароль для доступа к закрытому ключу:
  \begin{Verbatim}[commandchars=\\\[\]]
You need a Passphrase to protect your secret key.
  \end{Verbatim}

\item ``Пошуметь'' для генерации достаточного количества энтропии:
  \begin{Verbatim}[commandchars=\\\[\]]
We need to generate a lot of random bytes. It is a good 
idea to perform some other action (type on the keyboard,
move the mouse, utilize the disks) during the prime
generation; this gives the random number generator a
better chance to gain enough entropy.
  \end{Verbatim}
\end{enumerate}
\end{frame}

\begin{frame}[fragile]{Создание пары ключей -- 5}
  Результат:
  \begin{Verbatim}[commandchars=\\\[\]]
\$ gpg --list-secret-keys
/home/vip/.gnupg/secring.gpg
----------------------------
sec   2048R/F4A78166 2016-08-29
uid                  Vasiliy I. Pupkin <vip@example.ru>
ssb   2048R/BAED1502 2016-08-29
  \end{Verbatim}
\end{frame}

%%%

\subsection{Сертификат отзыва ключа}

\begin{frame}[fragile]{Сертификат отзыва ключа}
  \raisebox{-.30em}{\Large\HandRight}\hspace{.25em} Сертификат отзыва
  ключа (англ. \emph{revocation certificate}) -- Сертификат, который,
  будучи опубликованным, говорит о том, что связанный с ним ключ более
  не должен использоваться.\newline

  \begin{itemize}
  \item Сертификат должен храниться в безопасном месте (например, в
    сейфе.)
  \item Сертификат должен быть создан сразу после генерации приватного
    ключа.\newline
  \end{itemize}
  
  Случаи, при которых необходимо отзывать ключ:
  \begin{itemize}
  \item Забыт (утерян) пароль к ключу.
  \item Ключ был скомпроментирован.
  \item Ключ более не действителен.
  \item \ldots{}
  \end{itemize}
\end{frame}

\begin{frame}[fragile]{Генерация сертификата отзыва ключа -- 1}
\begin{Verbatim}[commandchars=\\\[\]]
\$ gpg --output revoke.asc --gen-revoke F4A78166

sec  2048R/F4A78166 2016-08-29
  Vasiliy I. Pupkin <vip@example.ru>

Create a revocation certificate for this key? (y/N) y
Please select the reason for the revocation:
  0 = No reason specified
  1 = Key has been compromised
  2 = Key is superseded
  3 = Key is no longer used
  Q = Cancel
(Probably you want to select 1 here)
Your decision? [\color[red]1]
\end{Verbatim}
\end{frame}

\begin{frame}[fragile]{Генерация сертификата отзыва ключа -- 2}
  Пример сертификата отзыва:\newline
\footnotesize
\begin{Verbatim}[commandchars=\\\[\]]
-----BEGIN PGP PUBLIC KEY BLOCK-----
Version: GnuPG v2
Comment: A revocation certificate should follow

iQEfBCABCAAJBQJXxAjjAh0CAAoJEHP7tQj0p4FmCpgIAJc8E6NriXKEl/AASVuL
LdiBrKcRIflk5me92mgKXd85GEFQGxIkwVziwr73CSzb4/As1pcfsxcF6WQw5173
vwTcml9fIzWTOJIr7qpj1uuTcnZlKagmMKWHh1sztmzX1YpOHV6hJ9t9xW0y1/js
5osETifAzh2k9b6NvAkniC3rqwL4DU7NwEkygkgjQgqmZFr0BedzgD5lmbcRRarP
llFLaqT7yGRZQbT9Nn1RBaPXgG7EvptNYuRYIdYEs2rxBNe/x5IHyWypp2Vacsrj
4PzXTxk2qpwcuRPHOcEtq0DxrdPwHC5fP1uyFXfVIQs/08pZvAdaq9dOrapaPvXs
UJ8=
=OkZL
-----END PGP PUBLIC KEY BLOCK-----
\end{Verbatim}
\normalsize
\end{frame}

\begin{frame}[fragile]{Отзыв ключа}
  \alert{Вы не должны выполнять отзыв ключа без причины!}\newline

  \begin{itemize}
  \item Импортировать сертификат:
    \footnotesize
\begin{Verbatim}[commandchars=\\\[\]]
\$ gpg --import revoke.asc
gpg: key F4A78166: "Vasiliy I. Pupkin <vip@example.ru>"
  revocation certificate imported
gpg: Total number processed: 1
gpg:    new key revocations: 1
gpg: 3 marginal(s) needed, 1 complete(s) needed, PGP trust model
gpg: depth: 0  valid:   3  signed:   0
  trust: 0-, 0q, 0n, 0m, 0f, 3u
\end{Verbatim}
    \normalsize
  \item Загрузить обновлённый ключ на сервер:
    \footnotesize
\begin{Verbatim}[commandchars=\\\[\]]
\$ gpg --keyserver gpg.mit.edu --send-keys F4A78166
\end{Verbatim}
    \normalsize
  \end{itemize}
\end{frame}


%%%

\section{Использование}

% \subsection{Шифрование}


\subsection{Электронная подпись}

\begin{frame}[fragile]{Электронная подпись файлов}
  Подписать файл можно следующим образом:
\begin{verbatim}
$ touch important-file.txt
$ gpg --detach-sign important-file.txt
\end{verbatim}

  Электронная подпись будет сохранена файле
  \texttt{important-file.txt.gpg}:
\begin{verbatim}
$ ls important-file*
important-file.txt important-file.txt.gpg
\end{verbatim}
\end{frame}


% \subsection{Сервера ключей}



\section{Сеть доверия}

\subsection{Сеть доверия}

\begin{frame}{Сеть доверия}
  \raisebox{-.30em}{\Large\HandRight}\hspace{.25em} Сеть доверия
  (англ. \emph{web of trust}) -- децентрализованная сеть, основанная
  на установлении однозначной связи между публичным ключом и
  владельцем.
\end{frame}

\begin{frame}{Встречи для подписи ключей}
  \raisebox{-.30em}{\Large\HandRight}\hspace{.25em} Встреча для
  подписи ключей (англ. \emph{key signing party}) -- мероприятие, на
  котором люди обмениваются отпечатками публичных ключей, с целью
  дальшейшей подписи публичных ключей друг друга и расширении сети
  доверия.
\end{frame}


\section{Примеры практического применения}

\subsection{Примеры практического применения}

\begin{frame}[fragile]{Шифрование почты}
  Существуют почтовые клиенты со встроенной поддержкой шифрования
  через GnuPG:
  \begin{itemize}
  \item KMail
  \item Evolution
  \item Thunderbird
  \item Mutt
  \item The Bat!
  \item Gnus
  \end{itemize}
\end{frame}




\subsection{Спасибо за внимание!}

\begin{frame}{Спасибо за внимание!}
  \large

  Эл. почта: \url{poptsov.artyom@gmail.com}

  \medskip

  Презентация и её ``исходники'' под лицензией Creative Commons:
  \url{github.com/artyom-poptsov/talks/tree/master/vcs} \\[10pt]

  Спасибо, что выслушали  :-) \\[30pt]

  \bigskip

  \huge Вопросы?
\end{frame}

\begin{frame}{Лицензия}
  Copyright \textcopyright 2015 Artyom V. Poptsov
  <poptsov.artyom@gmail.com> \newline

  Права на копирование других изображений, использованных в данной
  работе, принадлежат их владельцам. \newline

  Данная работа распространяется на условиях лицензии Creative Commons
  Attribution-ShareAlike 4.0 International:
  \url{https://creativecommons.org/licenses/by-sa/4.0/}
\end{frame}


\end{document}

%%% lection-0.tex ends here.
